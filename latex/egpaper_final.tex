\documentclass[10pt,twocolumn,letterpaper]{article}

\usepackage{iccv}
\usepackage{times}
\usepackage{epsfig}
\usepackage{graphicx}
\usepackage{amsmath}
\usepackage{amssymb}
\usepackage{algorithm}
\usepackage{algpseudocode}

\usepackage[breaklinks=true,bookmarks=false]{hyperref}

\iccvfinalcopy % *** Uncomment this line for the final submission

\begin{document}

%%%%%%%%% TITLE
\title{Dynamic Viewing\\CS280 Final Project}

\author{Rohan Chitnis\\
23464697\\
{\tt\small ronuchit@berkeley.edu}
\and
Alan Yao\\
23775635\\
{\tt\small alanyao@berkeley.edu}
\and
Alexander Chu\\
23460953\\
{\tt\small alex.p.chu@berkeley.edu}
}

\maketitle

\begin{abstract}
  Fast online tracking algorithms are fundamental to many real-world computer
  vision applications, such as video segmentation and human-computer interaction.
  We develop a multiscale sum-of-squared-displacements (SSD) template matching algorithm for
  tracking a face in a live video feed. Furthermore, we build invariance
  to rotation and scale directly into our system using linear interpolation on heuristic
  score functions. We evaluate performance through comparison with OpenCV's tracking
  algorithms based on Haar Cascades and show that our system performs TODO: how
  much better does it perform?
\end{abstract}

\section{Introduction}
3d graphics on monitors are unrealistic because they assume a center of projection
some distance away from the center of the screen. That is, they assume the viewer
is at a certain fixed position (usually in front of the middle of the screen) when
generating the projection. If the position of the viewer is known, however, the
center of projection can be set to match the viewer's point of view. As an analogy,
looking at the monitor will thus be much akin to looking through a window: the observed
scene undergoes certain projective transformations based on where the viewer is standing
relative to the window.\\

We develop an object recognition algorithm to track the position of any object
during a live webcam video feed. In order to achieve online tracking, which necessitates
quick computations, we use multiscale SSD template matching to determine the location of
the object. Additionally, we incorporate rotation and scale invariance, which allows us to
compute orientation and distance, and update the scene accordingly. We validate our
approach in a simulated computer game environment and TODO: describe the scenarios
we use for experimentation.

\section{Related Work}
TODO: can't do this till we have experiments

\section{Approach}

\subsection*{Multiscale SSD for Tracking}
SSD$(I, J)$ of two images $I$ and $J$ is computed as
$$\displaystyle\sum_{i, j}(I_{ij} - J_{ij})^{2}$$

At the heart of our system is a multiscale SSD template matcher, which operates as follows.
Initially, the user is asked to place his or her face inside an axis-aligned square window of size
$m$ by $m$, and the image inside this window is then used as a template for the online
portion of the tracking. A naive SSD-based approach would then proceed by sliding a window of
size $m$ by $m$ across the online image captured by the webcam, performing SSD at each window,
and then returning the window position corresponding to the minimum SSD. However, our initial
experimentation found this approach to be much too slow (enabling us to only reach about 10 fps,
which would have deproved upon introducing rotation and scale invariance).\\

In order to remedy this, we instead operate on multiple scales of both the online image and the
template image, by constructing a Gaussian pyramid and running the full sliding window for only
the topmost level of the pyramid. Then, once we have found the optimal window position which
minimizes SSD here, we have a small range within which we may search the next lowest level of
the pyramid. This continues until we have found the optimal window position for our original image.
This algorithm requires keeping track of the SSD search range at the current pyramid level.
Pseudocode for our implementation is described in Algorithm 1.\\

\begin{algorithm}
 \caption{Multiscale SSD sliding window for tracking.}
 \begin{algorithmic}[1]
  \Procedure{DetermineBestShift}{ImagePyramid, TemplatePyramid}
  \State $m \leftarrow$ window side length
  \State $N \leftarrow$ pyramid size
  \State wa $\leftarrow \lfloor m / 2^{N} \rfloor$
  \State tip $\leftarrow$ Topmost(ImagePyramid)
  \State ttp $\leftarrow$ Topmost(TemplatePyramid)
  \State r\_inds $\leftarrow$ (0, (tip.height - ttp.height) / wa,\\\hspace{57pt}0, (tip.width - ttp.width) / wa)
  \For {$i$ = $N$ \textbf{downto} 1}
     \State cip $\leftarrow$ ImagePyramid[$i$]
     \State ctp $\leftarrow$ TemplatePyramid[$i$]
     \State best\_i, best\_j $\leftarrow$ DoSSD(cip, ctp, $2^{i}$, r\_inds)
     \State r\_inds $\leftarrow$ (best\_i - 1, best\_i + 1,\\\hspace{73pt}best\_j - 1, best\_j + 1)
  \EndFor
  \State \Return best\_i, best\_j
  \EndProcedure
 \end{algorithmic}
  \begin{algorithmic}[1]
  \Procedure{DoSSD}{Image, Template, scaleAmt, r\_inds}
  \State $m \leftarrow$ window side length
  \State wa $\leftarrow \lfloor m / \text{scaleAmt} \rfloor$
  \State w $\leftarrow$ Template.width
  \State h $\leftarrow$ Template.height
  \For {$i$ = r\_inds[0] \textbf{to} r\_inds[1]}
    \For {$j$ = r\_inds[2] \textbf{to} r\_inds[3]}
    \State cand\_roi $\leftarrow$ Image[wa*i:wa*i+h,\\\hspace{122pt}wa*j:wa*j+w]
    \State calculate SSD(Template, cand\_roi)
    \EndFor
  \EndFor
  \State \Return $i$, $j$ having minimum SSD
  \EndProcedure
 \end{algorithmic}
\end{algorithm}

Some clarifications of Algorithm 1 are in order. The variable \texttt{wa}
represents the window size at the current pyramid level. The variable \texttt{r\_inds}
is a tuple of the form (start $i$, end $i$, start $j$, end $j$), and it keeps
track of the region of the online image within which the SSD must be computed
at this level of the pyramid, in units of window size (this is why it gets
multiplied by \texttt{wa} in the calculation of the candidate region, at line 8 of
DoSSD). Note that the algorithm assumes a downsampling factor of 2 at each level
of the two pyramids.

\subsection*{Building Invariance}
Algorithm 1, multiscale SSD, only provides translational invariance between the
template image and the online image. However, this is limited in utility, so we
extend the algorithm to provide rotational and scale invariance, and in the
process develop a method of determining the orientation and scale both quickly
and accurately. If one had infinite computational resources, one would discretize
each of these quantities into a large number of buckets and run Algorithm 1 for
the template at each quantity. For example, one could run Algorithm 1 on the template
image rotated each degree from -45 to 45, and take the orientation which minimizes SSD.
However, this is clearly not scalable down to a practical setting.\\

Instead, we propose the following approach. For orientation, we run Algorithm 1
on the template image rotated to a small number of preset orientations, then based on the SSD value
of each of these, we perform linear interpolation to estimate the current orientation.
The interpolation weights are based on the SSD values: since a lower SSD means we want
to give this rotation amount a higher weight, we use the multiplicative inverse of the
SSD for the interpolation weights. The general formula then becomes
$$rot = \displaystyle\sum_{k}(\dfrac{1.0 / SSD_{k}}{\sum_{k}(1.0 / SSD_{k})} * rots[k])$$
where $rots$ is an array that holds all the rotation amounts in degrees,
the summation is over the indices of this array, and $SSD_{k}$ represents the SSD value
corresponding to the rotation amount $rots[k]$.

Similarly,
$$scale = \displaystyle\sum_{k}(\dfrac{1.0 / SSD_{k}}{\sum_{k}(1.0 / SSD_{k})} * scale[k])$$

Note that this requires storage of online pyramids and template pyramids corresponding
to each of the rotation amounts and scale amounts.

\subsection*{Rendering Engine Updates}
TODO: need to put stuff about calibration and how you get focal length from
it here, including all the assumptions about average face width that we made,
also need to put stuff about kalman filter smoothing here, also all the update
equations given the orientation, scale, and window position

NOTE: avoid using specific numbers here. describe the specific numbers in
the next section: Setting

\section{Experiments}

\subsection*{Setting}
We evaluate performance in a simulated computer game environment loosely based
on Minecraft. We feed our webcam image through Algorithm 1 for each rotation
value and each scale value to determine the orientation and distance amounts,
and the window position which minimizes SSD for these amounts. Then, we use this
information to update the rendering engine as previously described. In our experimentation,
we used rotation amounts of -45, 0, and 45 degrees, and scale amounts of 0.5, 1, and 1.5,
for the interpolation. Furthermore, we used a pyramid size of 3 (where the bottom
layer is simply the original image) and a downsampling factor of 2. Our window size
was 8 pixels.

Our development environment was Python 2.7, and our rendering engine was Pyglet.
TODO: add screenshots showing that things work

One issue we noted is that the SSD algorithm is not good at detecting whether the template
image is in the online image or not (it is only good at tracking the image). For this
purpose, we run an SVM which tells us whether the online image contains the template image,
and if it doesn't, smoothly update the rendering to avoid jerkiness.

\subsection*{Results}
TODO: describe numeric results

\subsection*{Evaluation}
TODO: discuss comparison with opencv

\section{Conclusion}
The need for fast tracking in an online setting is well-understood and has utility
in several applications of computer vision. We developed and evaluated performance
of an algorithm which uses multiscale SSD template matching and offers invariance to
capture orientation and distance information. We further demonstrated the applicability of our
algorithm by illustrating real-time performance in a simulated computer game environment,
and compared its performance to that of OpenCV.

{\small
\bibliographystyle{ieee}
\bibliography{egbib}
}

\end{document}
